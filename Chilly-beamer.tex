% !TEX encoding = UTF-8 Unicode
% !TEX TS-program = XeLaTeX
% !TEX spellcheck = English
% !TEX pdfSinglePage



{\let~\catcode~13=13\def^^M{^^J}~` 12~`\%12~`~12\xdef\asciiart{
                                                                                  
            ...                         .          ..       ..                    
         xH88"`~ .x8X      .uef^"      @88>  x .d88"  x .d88"    ..               
       :8888   .f"8888Hf :d88E         %8P    5888R    5888R    @L                
      :8888>  X8L  ^""`  `888E          .     '888R    '888R   9888i   .dL        
      X8888  X888h        888E .z8k   .@88u    888R     888R   `Y888k:*888.       
      88888  !88888.      888E~?888L ''888E`   888R     888R     888E  888I       
      88888   %88888      888E  888E   888E    888R     888R     888E  888I       
      88888 '> `8888>     888E  888E   888E    888R     888R     888E  888I       
      `8888L %  ?888   !  888E  888E   888E    888R     888R     888E  888I       
       `8888  `-*""   /   888E  888E   888&   .888B .  .888B .  x888N><888'       
         "888.      :"   m888N= 888>   R888"  ^*888%   ^*888%    "88"  888        
           `""***~"`      `Y"   888     ""      "%       "%            88F        
                               J88"                                   98"         
                               @%                                   ./"           
                             :"                                    ~`             
                                                                                  
}}\makeatletter



\documentclass[14pt,aspectratio=1610]{beamer}
	\beamertemplatenavigationsymbolsempty
	\setbeamercovered{transparent}
	\setbeamersize{text margin left=3mm,text margin right=3mm}
	\overfullrule1em
	\def\pp{\pause\par}
	\advance\parskip\fill
	\def\CMH{\gdef\beamer@currentmode{handout}}
	\def\CMB{\gdef\beamer@currentmode{beamer}}

\catcode`激13 \def激#1{\lccode`~`#1\lowercase{\catcode`#113\def~}}
	激色#1!#2#3#4#5#6#7{\definecolor{#1}{HTML}{#2#3#4#5#6#7}}
	% https://marketing.illinois.edu/brand/design/color
	色Illini Orange!FF552E		色Altgeld!DD3403				色Illini Blue!13294B		
	色Alma! 1E3877色Industrial! 1D58A7色Arches! 009FD4色Cloud! F8FAFC色Heritage! F5821E
	色Alma2!4D69A0色Industrial2!5783BC色Arches2!7FC3E1色Cloud2!E8E9EB色Heritage2!E56E15
	色Alma3!849BC1色Industrial3!90AED5色Arches3!A6D7EB色Cloud3!DDDEDE色Heritage3!CE5E11
	色Alma4!AFC7DB色Industrial4!CAD9EF色Arches4!D2EBF5色Cloud4!D2D2D2色Heritage4!B74D04

	\setbeamercolor{normal text}{bg=Illini Blue,fg=Cloud}
	\setbeamercolor{structure}{fg=Illini Orange}
	\setbeamercolor{alerted text}{fg=Heritage}
	\setbeamercolor{example text}{fg=Arches}

\usepackage{xurl}
	\hypersetup{pdfsubject=94A24; 68P30; math.IT}
	\hypersetup{colorlinks,linkcolor=Heritage,citecolor=Altgeld,urlcolor=Arches}

\usepackage{mathtools}

\usepackage[warnings-off={mathtools-colon,mathtools-overbracket}]{unicode-math}
	\setmainfont{Times}
	\setsansfont{HelveticaNeue-Light}
	\setmonofont{Menlo}
	\setmathfont[sans-style=literal]{texgyrepagella-math.otf}
	
	\DeclareMathOperator*\argmax{argmax}
	\DeclareMathOperator*\mybest{do-my-best}
	\DeclareMathOperator\hwt{hwt}
	\DeclareMathOperator\poly{poly}
	\def\P{P_\mathrm e}
	
	\def\({\bigl(}	\def\){\bigr)}	激({\Bigl(}		激){\Bigr)}		
	激[{\bigl[}		激]{\bigr]}		激「{\Bigl[}		激」{\Bigr]}		
	激{{\bigl\{}	激}{\bigr\}}	激『{\Bigl\{}	激』{\Bigr\}}	
	激【{\lvert}		激】{\rvert}	
	
	激㏒{\log}		激㏑{\ln}		
	激ˆ{\hat}		激¯{\bar}		
	激|{\mid}		激:{\colon}		激;{\mathrel;\nobreak}		
	激÷{\frac}		激√{\sqrt}		激¬{\limits}		
	激†#1†{{\text{#1}}}				
	激⋆{\raisebox{1ex}{$\star$}}
	
	\def\J#1{^{(#1)}}
	\def\W#1{W\J{#1}}
	\def\WW#1#2{(\W{#1})\J{#2}}
	\def\WWW#1#2#3{\(\WW{#1}{#2}\)\J{#3}}
	\def\WWWW#1#2#3#4{(\WWW{#1}{#2}{#3})\J{#4}}
	\def\Q#1{Q\J{#1}}
	\def\QQ#1#2{(\Q{#1})\J{#2}}
	\def\QQQ#1#2#3{\(\QQ{#1}{#2}\)\J{#3}}
	\def\G#1{G\J{#1}}
	\def\GG#1#2{\G{#1#2}}

\usepackage{tikz,tikz-cd}
	\pgfmathsetseed{32652}\let\PMP\pgfmathparse\def\PMR{\pgfmathresult}
	\let\PMS\pgfmathsetmacro\let\PMT\pgfmathtruncatemacro\let\PMD\pgfmathdeclarefunction
	% https://tex.stackexchange.com/q/420034/
	\PMD*{axis_height}0{\begingroup\pgfmathreturn.25em\endgroup}
	\PMD*{rule_thickness}0{\begingroup\pgfmathreturn.06em\endgroup}
	\tikzset{
		every picture/.style={cap=round,join=round,line width=rule_thickness},
		% https://tex.stackexchange.com/q/146908/
		alt/.code args={<#1>#2#3}{\alt<#1>{\pgfkeysalso{#2}}{\pgfkeysalso{#3}}},
		uncover/.style={alt=<#1>{}{opacity=.15}},
	}
	\def\channeltree#1<#2:#3v#4v#5;{
		\edef\d{#1}\edef\dept{#2}\edef\zbottom{#3}\edef\z{#4}\edef\ztop{#5}
		\ifnum\d<\dept
		\ifdim\z pt<\ztop pt
		\ifdim\z pt>\zbottom pt{
			\advance\c@beamerpauses1
			\edef\dplus{\the\numexpr\d+1}
			{
				\PMS\zup{\z*(256-\z)/128}
				\onslide<.->{\draw(\d,\z/32)--(\dplus,\zup/32);}
				\channeltree\dplus<\dept:\zbottom v\zup v\ztop;
			}
			{
				\PMS\zdown{\z^2/128}
				\onslide<.->{\draw(\d,\z/32)--(\dplus,\zdown/32);}
				\channeltree\dplus<\dept:\zbottom v\zdown v\ztop;
			}
		}\fi\fi\fi
		\onslide<.->{\fill(\d,\z/32)circle(1pt);}
	}

\usepackage{pgfplotstable,booktabs,colortbl}
	\pgfplotsset{compat/.cd,show suggested version=false,=1.17}
	\pgfplotstableset{
		every head row/.style={before row=\toprule,after row=\midrule},
		every last row/.style={after row=\bottomrule},string type,
	}
	\def\arraystretch{1.44}

\title[Complexity \& 2-Moment of Communication]{
	Complexity and Second Moment of	\\
	the Mathematical Theory of Communication
}
\author[H-P\ Wang]{Hsin-Po WANG}
\institute{Department of Mathematics, University of Illinois at Urbana--Champaign}
\date[2021-04]{2021-04-01 PhD Defense Presentation}

\begin{document}
\message{\asciiart}
\makeatletter

\def\linkfil#1{\vbox to#1cm{\hbox{.}\vfil\hbox to3mm{\hfil.}}\vfil}
\defbeamertemplate*{sidebar left}{thinbold}{
	\pgfsetfillopacity0
	\hyperlinkframeendprev{\linkfil1}
	\hyperlinkslideprev{\linkfil3}
	\hyperlinkslidenext{\linkfil3}
	\hyperlinkframestartnext{\linkfil1}
	\pgfsetfillopacity1\vfilneg\vfilneg
}

\frame\maketitle

\defbeamertemplate*{sidebar right}{thinbold}{
	\tikz[remember picture,overlay,x=3mm,y=\paperheight]{\footnotesize
		\PMS\olfrac{\insertoverlaynumber/(\insertframeendpage+1-\insertframestartpage)}
		\path[save path=\stare,yscale={1/max(\insertmainframenumber-1,1)}]
			(0,0)-|(-1,1-\insertframenumber)-|+(\olfrac,1)-|cycle;
		\tikzset{banner/.pic={\node at(-.55,-.5)[rotate=-90]
			{\beamer@shorttitle~~\beamer@shortdate~~\beamer@shortauthor};}}
		\fill[use path=\stare,Alma3]pic[Alma4]{banner};
		\clip[use path=\stare]pic[Illini Blue]{banner};
	}
}

\CMB

\frame{{Noisy channel}
	The sender inputs  $X₁³²=\texttt{11001001 00001111 11011010 10100010}$.
	\pp
	The channel outputs $Y₁³²=\texttt{1--01-01 ----1--- -101---0 --0--0-0}$.
}

\frame{{Noisy channel}
	Sender inputs $X₁³²∈𝔽_q³²$, where $𝔽_q$ is called input alphabet.	\\
	WLoG, we may assume $𝔽_q$ is a finite field [new idea].
	
	The channel outputs $Y₁³²$ according to transition probabilities
	$P\{Y_j=y|X_j=x\}=W(y|x)$ independently for each $j$.
}

\frame{{Noisy-channel coding}
	The encoder inputs $X₁³²∈ℬ⊊𝔽_q^{32}$ into a channel.	\\
	$ℬ$ is a \emph{block code} (sometimes a \emph{codebook}) of block length $N=32$.
	\pp
	The channel outputs $Y₁³²$ according to $W(y|x)$.
	\pp
	The decoder, seeing $Y₁³²=y₁³²$, maximizes the posterior probability	\\
	$ˆX₁³²(y₁³²)≔{\alt<+(1)>{\color{alerted text.fg}\mybest}
		\argmax¬_{x₁³²∈ℬ}}P\{X₁³²=x₁³²|Y₁³²=y₁³²\}$.
}

\frame{{Noisy-channel coding theorem}
	Channel capacity $C≔\sup¬_{\!X∼Q\!}I(X;Y)$ (supremum over input distribution).	\\
	Block length is $N$.	\\
	Error probability is $\P≔P\{ˆX₁^N≠X₁^N\}$.	\\
	Code rate is $R≔㏑【ℬ】\div㏑【𝔽_q^N】$.	\hfill(recall that $ℬ⊂𝔽_q^N$)	
	\pp
	[Shannon 1948]	\\
	\emph{
		One can find block codes $ℬ$ such that $\P→0$ and $R→C$ as $N→∞$.	\\
		(And $C$ is the greatest number that allows this to happen.)
	}
}

\frame{{2nd-order term of coding}
	How fast do error probability $\P$ and code rate $R$ converge to $0$ and $C$	\\
	as block length $N→∞$? Characterize functions ``$\P(N)$'' and ``$R(N)$''.
	\pp
	When $R$ is fixed, $\P≈e^{-N}$; or equivalently, $-㏑\P≈N$.	\\
	Fano \cite{Fano61},
	Gallager \cite{Gallager65},
	Shannon--Gallager--Berlekamp \cite{SGB67},
	\cite{Gallager68},
	\cite{Gallager73},
	Blahut \cite{Blahut74},
	Barg--Forney \cite{BF02},
	Fàbregas--Land--Martinez \cite{FLM11},
	Domb--Zamir--Feder \cite{DZF16}.
	\pp
	When $\P$ is fixed, $R≈C-N^{-1/2}$; or equivalently, $(C-R)^{-2}≈N$.	\\
	Wolfowitz \cite{Wolfowitz57},
	Weiss \cite{Weiss60},
	Dobrushin \cite{Dobrushin61},
	Strassen \cite{Strassen62},
	Baron--Khojastepour--Baraniuk \cite{BKB04},
	Hayashi \cite{Hayashi09},
	Polyanskiy--Poor--Verdu \cite{PPV10}.
}


\frame{{Joint 2nd-order term of coding}
	When both $R$ and $\P$ vary, $(\P,R)≈(e^{-N^π},C-N^{-ρ})$ where $π+2ρ=1$;	\\
	or equivalently, $(-㏑\P)(C-R)^{-2}≈N$.	\\
	Altuğ--Wagner \cite{AW10},
	Polyanskiy--Verdú \cite{PV10},
	\cite{AW14},
	Hayashi--Tan \cite{HT15}.
	\pp
	This is a two-sided bound:	\\
	A code $ℬ$ exists such that $(-㏑\P)(C-R)^{-2}≈N$.	\\
	No code $ℬ$ exists such that $(-㏑\P)(C-R)^{-2}≫ N$.
	\pp
	Block length $N$ is your income;	\\
	invest in error probability $\P$ or in code rate $R$ or in both.
}

激分#1#2{÷{#1\rule{0ex}{1.5ex}}{#2\rule[-1ex]{0ex}{0ex}}}
\pgfplotstableread{
	Par	Paradigm						&	{Random variable}					
	LLN	{law of large numbers}			&	¯X→μ								
	LDP	{large deviation principle}		&	ℙ\{¯X-μ>x\}≈e^{-nI(x)}				
	CLT	{central limit theorem}			&	¯X-μ∼𝒩(0,σ/√n)						
	MDP	{moderate deviation principle}	&	分{-㏑ℙ\{¯X-μ>γ_nx\}}{γ_n²}≈nI(x)	
}\tableTrinity
\pgfplotstableread{
	Par	&	{Random code}		{Polar code}				
	LLN	&	(\P,R)→(0,C)		(\P,R)→(0,C)				
	LDP	&	\P≈e^{-N}			\P≈e^{-N^{0.99}}			
	CLT	&	C-R≈N^{-1/2}		C-R≈N^{-0.49}				
	MDP	&	分{-㏑\P}{(C-R)²}≈N	分{-㏑\P}{(C-R)²}≈N^{0.99}	
}\tableCoding
\pgfplotstablecreatecol[create col/copy column from table=\tableCoding{Random code}]
	{Random code}\tableTrinity
\pgfplotstablecreatecol[create col/copy column from table=\tableCoding{Polar code}]
	{Polar code}\tableTrinity
\pgfplotstablemodifyeachcolumnelement{Paradigm}\of\tableTrinity\as\cell
	{\edef\cell{\noexpand\footnotesize\unexpanded\expandafter{\cell}}}
\pgfplotstablemodifyeachcolumnelement{Random variable}\of\tableTrinity\as\cell
	{\edef\cell{$\unexpanded\expandafter{\cell}$}}
\pgfplotstablemodifyeachcolumnelement{Random code}\of\tableTrinity\as\cell
	{\edef\cell{$\unexpanded\expandafter{\cell}$}}
\pgfplotstablemodifyeachcolumnelement{Polar code}\of\tableTrinity\as\cell
	{\edef\cell{$\unexpanded\expandafter{\cell}$}}

\frame{{2nd-order term analog}
	$$\pgfplotstabletypeset[
		columns={Paradigm,Random variable,Random code},
		columns/Random variable/.style={column type={>{\onslide<2->}c}},
		columns/Random code/.style={column type={>{\onslide<3>}c}},
	]\tableTrinity$$
}

\frame{{However...}
	Achievability via random coding assumes exponential complexity	\\
	due to the usage of the maximum a posteriori decoder $\argmax¬_{x₁³²∈ℬ}$.
	\pp
	Goal:
	Comparable performance, but with a low-complexity $\mybest¬_{x₁³²∈ℬ}$.
}

\frame{{2nd-order term goal}
	$$\pgfplotstabletypeset[
		columns={Paradigm,Random code,Polar code},
		columns/Polar code/.style={
			column type={>{\onslide<2>}c},column name=Low-complexity code},
	]\tableTrinity$$
	
	\hfill\onslide<2>($π,ρ>0$ and $π+2ρ<1)$
}

\pgfplotstableread{
	Par		BEC			SBDMC		p-ary	q-ary	finite		BDMC	a-finite
	LLN		Arikan09	Arikan09	STA09	STA09	STA09		SRDR12	w		
	LDP⋆	AT09		AT09		STA09	MT10	Sasoglu11	HY13	w		
	CLT⋆	KMTU10		HAU14		BGNRS18	w		w			w		w		
	MDP⋆	GX15		GX15		BGS18	w		w			w		w		
	LDP		KSU10		KSU10		w		w		w			w		w		
	CLT		FHMV18		GRY20		w		w		w			w		w		
	MDP		w			w			w		w		w			w		w		
}\tableRefarray
\def\assigncontent#1{\pgfkeyssetvalue{/pgfplots/table/@cell content}{#1}}
\def\decodecontent#1#2\relax{
	\if\pgfplotstablecol0	\assigncontent{#1#2}
	\else\if#1w				\assigncontent{??}
	\else					\assigncontent{\footnotesize\cite{#1#2}}
	\fi\fi
}
\frame{{Polar coding}
	$$\pgfplotstabletypeset[
		columns/Par/.style={column type={>{\onslide<1->}c}},
		columns/BEC/.style={column type={>{\onslide<2->}c}},
		columns/SBDMC/.style={column type={>{\onslide<2->}c}},
		columns/p-ary/.style={column type={>{\onslide<3->}c}},
		columns/q-ary/.style={column type={>{\onslide<4->}c}},
		columns/finite/.style={column type={>{\onslide<4->}c}},
		columns/BDMC/.style={column type={>{\onslide<4->}c}},
		columns/a-finite/.style={column type={>{\onslide<5->}c}},
		assign cell content/.code={\decodecontent####1\relax}
	]\tableRefarray$$
}

\frame{{Polar coding road map}
	\textcolor{example text.fg}{Channel transformation}
		manipulates channels.
	
	\textcolor{example text.fg}{Channel tree}
		is the result of recursive channel transformation.
	
	\textcolor{example text.fg}{Channel parameter}
		measures the channels and keep track of the tree.
	
	\textcolor{example text.fg}{Channel process}
		is a syntax candy paraphrasing the tree.
	
	\textcolor{example text.fg}{Channel polarization}
		is a phenomenon that channels become extreme.
}

\frame{{Channel transformation}
	Consider a channel $W=(X|Y)$, where input is $X$, output is $Y$.	\\
	Make two i.i.d.\ copies $(X₁|Y₁)$ and $(X₂|Y₂)$.
	\pp
	$\W1≔(X₁-X₂|Y₁²)$;	\\
	$\W2≔(X₂|(X₁-X₂)Y₁²)$.	\hfill(juxtaposition is tuple concatenation)
}

\frame{{Channel transformation (other kernel)}
	$U₁²$: two free variables; $G$: a $2×2$ matrix (called kernel);	\\
	$X₁²≔U₁²·G$: matrix multiplication; channels generate $Y₁²$ given $X₁²$.
	
	$\W1≔(U₁|Y₁²)$;	\\
	$\W2≔(U₂|U₁Y₁²)$	\hfill(juxtaposition is tuple concatenation).
}

\frame{{Channel transformation (larger kernel)}
	$U₁^ℓ$: many free variables; $G$: an $ℓ×ℓ$ matrix kernel;	\\
	$X₁^ℓ≔U₁^ℓ·G$; channels generate $Y₁^ℓ$ given $X₁^ℓ$.
	\pp
	$\W1≔(U₁|Y₁^ℓ)$;	\\
	$\W2≔(U₂|U₁Y₁^ℓ)$;	\\
	$\W3≔(U₃|U₁²Y₁^ℓ)$;	\\
	$\phantom{\W{ℓ-1}}⋮$	\\
	$\W{ℓ-1}≔(U_ℓ|U₁^{ℓ-2}Y₁^ℓ)$;	\\
	$\Wℓ≔(U_ℓ|U₁^{ℓ-1}Y₁^ℓ)$.
	\pause
	\tikz[x=1em,y=-1em,overlay,shift={(7,-6)}]{
		\draw
			(0,.5)rectangle node(G){$·G$}(3,5.5)
			foreach\j in{1,...,5}{
				(-2,\j)node(U\j){$U_\j$}(U\j)-|(0,3)
				(5,\j)node(X\j){$X_\j$}(3,3)|-(X\j)
				(8,\j)node(Y\j){$Y_\j$}(X\j)--(Y\j)
			}
		;
	}
}

\frame{{Channel tree}
	Channel $W$ grows $\W1,\W2,…,\Wℓ$.
	\pp
	For each $i$, channel $\W i$ grows $\WW i1,…,\WW iℓ$.
	\pp
	For each $j$, channel $\WW ij$ grows $\WWW ij1,…,\WWW ijℓ$.
	\pp
	$$\tikz[
		grow=right,
		level/.style={
			level distance=2em*####1,sibling distance=8em/2^####1,nodes={scale=7/8}
		}
	]{
		\node{$W$}
		child foreach\i in{1,2}{
			node{$\W\i$}
			child foreach\j in{1,2}{
				node{$\WW\i\j$}
				child foreach\k in{1,2}{
					node{$\WWW\i\j\k$}
				}
			}
		}
	}$$
}

\frame{{Dynamic kernel [new idea*]}
	Channel $W$ grows $\W1,\W2,…,\Wℓ$ using $G$.
	\pp
	For each $i$, channel $\W i$ grows $\WW i1,…,\WW iℓ$ using $\G i$.
	\pp
	$∀j$, channel $\WW ij$ grows $\WWW ij1,…,\WWW ijℓ$ using $\GG ij$.
	\pp
	$$\tikz[
		grow=right,
		level/.style={
			level distance=2em*####1,
			sibling distance=8em/2^####1,
			nodes={scale=7/8}
		}
	]{
		\node{$W$}
		child foreach\i in{1,2}{
			node{$\W\i$}
			child foreach\j in{1,2}{
				node{$\WW\i\j$}
				child foreach\k in{1,2}{
					node{$\WWW\i\j\k$}
				}
			}
		}
	}$$
}

%	$W$ grows $\W1$.
%	$\W1$ grows $\WW11$ and $\WW12$;
%	$\W2$ grows $\WW21$ and $\WW22$.
%	$\WW11$ grows $\WWW111$ and $\WWW112$;
%	$\WW12$ grows $\WWW121$ and $\WWW122$;
%	$\WW21$ grows $\WWW211$ and $\WWW212$;
%	$\WW22$ grows $\WWW221$ and $\WWW222$.

\frame{{Channel parameter ($ℓ=2$ and $n=3$ exmaple)}
	Block length $N=ℓ^n$; for instance $8=2³$.
	\pp
	Select indices $𝒥⊆\{1,…,ℓ\}^n$; for instance $\{122,212,221,222\}⊆\{1,2\}³$.	\\
	Code rate $R=【𝒥】/N=4/8$ (nontrivial, due to implementation details).
	\pp
	Error probability $\P≤∑¬_{ijk∈𝒥}H(\WWW ijk)$ (nontrivial, due to details);	\\
	$H(X|Y)$ is conditional entropy with base-$q$ logarithm.
}

\frame{{It suffices to understand}
	$H(W),H(\W i),H\(\WW ij\),H(\WWW ijk),H(\WWWW ijkl),…$\kern-2em
	\pp
	Block length $N$ will be $ℓ^†where we stop†$.
	\pp
	Code rate $R$ will be the fraction of small $H$-values.
	\pp
	Block error probability $\P$ will be $∑¬_†those†$ small $H$-values.
}

\frame{{Channel process (a powerful syntax candy)}
	$𝘞₀≔W$.	\\
	$𝘞_{n+1}≔𝘞_n^{(𝘑_{n+1})}$, where
	$𝘑_{n+1}∈\{1,2,…,ℓ\}$ i.i.d.\ uniform branch chooser.
	\pp
	$𝘏_n≔H(𝘞_n)$.
	\pp
	Decide depth $n$, then block length is $N=ℓ^n$.	\\
	Decide threshold $θ$, then code rate is $R=𝘗\{𝘏_n<θ\}$.	\\
	Error probability is $\P<∑†small †𝘏_n<∑θ=RNθ≤Nθ$.
}

\frame{{Channel polarization}
	$𝘏_n≔H(𝘞_n)$ is a martingale. (Invoke Doob's martingale convergence)	\\
	$𝘏_n→𝘏_∞$ a.e.\ as $n→∞$; turns out $𝘏_∞∈\{0,1\}$ and $𝘗\{𝘏_∞=1\}=𝘏₀$.
	
	\onslide<+>{}
	$$\tikz{
		\path(0,0)circle(2pt)(0,4)circle(2pt);
		\channeltree0<6:0v49v128;
		\onslide<.(7)->{
			\draw[alerted text.fg](6,1/4)--+(1,0)node[right]{threshold $θ$};}
	}$$
}


\frame{{It suffices to understand}
	\par
	{\LARGE$𝘗\{𝘏_n<†threshold†\}>C-†gap†$.\par}
	\pp
	Goal: $𝘗{𝘏_n<e^{-ℓ^{πn}}}>C-ℓ^{-ρn}$ for large $n$, where $π+2ρ<1$.	\\
	Then: $N=ℓ^n$ and $\P<Ne^{-N^π}≈e^{-N^π}$ and $R>C-N^{-ρ}$ .
}

\frame{{Proof outline}
	\textcolor{example text.fg}{Local LDP behavior:}
	$Z(\W j)≤ℓe^{qZ(W)ℓ}(qZ(W))^{⌈j^2/3ℓ⌉}$,	\\
	where $Z$ is a parameter such that $Z≤q³√H$ and $H≤q³√Z$.
	
	\textcolor{example text.fg}{Local CLT behavior:}
	$∑¬_{j=1}^ℓh(H(\W j))<4ℓ^{1/2+α}$,	\\
	where $α=㏑(㏑ℓ)/㏑ℓ$ and $h(z)≔\min(z,1-z)^α$.
	
	\textcolor{example text.fg}{Global MDP behavior:}
	$𝘗{𝘏_n<e^{-ℓ^{πn}}}>C-ℓ^{-ρn}$,
	where $π+2ρ<1$, given local LDP and local CLT behaviors.
}

\frame{{Local LDP behavior 1/3}
	Want to prove $Z(\W j)≤ℓe^{qzℓ}(qz)^{⌈j^2/3ℓ⌉}$ where $z≔Z(W)$.
	\pp
	Fundamental theorem of polar:
	$Z(\W j)≤∑¬_{u_{j+1}^ℓ∈𝔽_q^{ℓ-j}}z^{\hwt(0₁^{j-1}1_ju_{j+1}^ℓ·G)}$.	\\
	RHS is the weight enumerator of a coset code.
	\pp
	$\W1≔(U₁|Y₁^ℓ)$;	\\
	$\W2≔(U₂|U₁Y₁^ℓ)$;	\\
	$\phantom{\W2}⋮$	\\
	$\Wℓ≔(U_ℓ|U₁^{ℓ-1}Y₁^ℓ)$.
	\tikz[x=1em,y=-1em,overlay,shift={(6,-5)}]{
		\draw
			(0,.5)rectangle node(G){$·G$}(3,5.5)
			foreach\j in{1,...,5}{
				(-2,\j)node(U\j){$U_\j$}(U\j)-|(0,3)
				(5,\j)node(X\j){$X_\j$}(3,3)|-(X\j)
				(8,\j)node(Y\j){$Y_\j$}(X\j)--(Y\j)
			}
		;
	}
}

\frame{{Local LDP behavior 2/3}
	Want $∑¬_{u_{j+1}^ℓ}z^{\hwt(0₁^{j-1}1_ju_{j+1}^ℓ·G)}
		≤ℓe^{qzℓ}(qz)^{⌈j^2/3ℓ⌉}$ for some $G$.
	\pp
	Draw random $𝔾$ instead; $𝔼[†LHS†]=q^{-j}(1+(q-1)z)^ℓ≤q^{-j}(1+qz)^ℓ$.
	\pp
	Compare $(qz)^w$-coefficients:
	$q^{-j}\binomℓw$ vs $ℓ÷{ℓ^{w-⌈j^2/3ℓ⌉}}{(w-⌈j^2/3ℓ⌉)!}$.
	\pp
	Simplify: $2^{-j}\binomℓ{⌈j^2/3ℓ⌉}\binom{ℓ-⌈j^2/3ℓ⌉}{w-⌈j^2/3ℓ⌉}$
	vs $ℓ\binomℓ{w-⌈j^2/3ℓ⌉}$.
}

\PMD{h2o}1{% := h2(sin(#1)^2)
	\PMS\sin{sin(#1)}\PMS\cos{cos(#1)}%
	\ifdim\sin pt=0pt%
		\PMP{0}%
	\else\ifdim\cos pt=0pt%
		\PMP{0}%
	\else%
		\PMP{(-\sin*ln(\sin)*\sin-\cos*ln(\cos)*\cos)*2.88539}% 2/ln(2)
	\fi\fi%
}
\frame{{Local LDP behavior 3/3}
	Boils down to $2^{-j}\binomℓ{⌈j^2/3ℓ⌉}$ vs $ℓ$;
	ignore $⌈⌉$ and $ℓ$; compare $\binomℓ{j^2/3ℓ}$ vs $2^j$.
	\pp
	$\binomℓd≈2^{ℓh_2(d/ℓ)}$ for $d=Θ(ℓ)$. (Large deviations theory.)	\\
	Hence $h_2(j^2/3ℓ^2)$ vs $j$, which becomes $√{3x}$ vs $h_2(x)$.
	\pp
	$$\tikz[scale=4]{
		\path(0,0)--(1,1);
		\tikzset{
			shift={(.2,.8)},
			scale={1.2^max(0,\insertoverlaynumber-\c@beamerpauses)},
			shift={(-.2,-.8)},
		}
		\draw[overlay]plot[domain=0:90,samples=90]({sin(\x)^2},{h2o(\x)});
		\draw[overlay]plot[domain=0:1,samples=20](\x^2/e,\x);
		\only<+(10)>{}
	}$$
}

\frame{{Local CLT behavior 1/4}
	Want to prove $∑¬_{i=1}^ℓh(H(\W i))<4ℓ^{1/2+α}$,	\\
	where $α=㏑(㏑ℓ)/㏑ℓ$ and $h(z)≔\min(z,1-z)^α$.
	\pp
	Break into three segments
	$\begin{cases}
		∑¬_{i=⌈H(W)+ℓ^{-1/2+α}⌉+1}^ℓh(H(\W i))<ℓ^{1/2+α},	\\
		∑¬_{i=⌊H(W)-ℓ^{-1/2+α}⌋}^{⌈H(W)+ℓ^{-1/2+α}⌉}h(H(\W i))<2ℓ^{1/2+α}, \\
		∑¬_{i=1}^{⌊H(W)-ℓ^{-1/2+α}⌋-1}h(H(\W i))<ℓ^{1/2+α}.
	\end{cases}$
	\pp
	\tikz[overlay,shift={(1,-1)}]{
		\draw(0,0)[domain=0:3.1748]plot(\x^3/16,\x)plot(4-\x^3/16,\x);
		\fill foreach\y in{.1,.2,.3,.5,.8,1.3,2.1}{(\y^3/16,\y)circle(2pt)};
		\fill foreach\y in{.1,.2,.4,.8,1.6}{(4-\y^3/16,\y)circle(2pt)};
	}
}

\frame{{Local CLT behavior 2/4}
	Want $∑¬_{i=j+1}^ℓh(H(\W i))<ℓ^{1/2+α}$, where $j≔⌈H(W)+ℓ^{-1/2+α}⌉$.
	\pp
	Jensen LHS; want to show $(ℓ-j)h(÷1{ℓ-j}∑¬_{i=j+1}^ℓH(\W i))<ℓ^{1/2+α}$.
	\pp
	$\phantom{\W{ℓ-2}}⋮$	\\
	$\W{ℓ-2}≔(U_{ℓ-2}|U₁^{ℓ-3}Y₁^ℓ)$,	\\
	$\W{ℓ-1}≔(U_{ℓ-1}|U₁^{ℓ-2}Y₁^ℓ)$,	\\
	$\Wℓ≔(U_ℓ|U₁^{ℓ-1}Y₁^ℓ)$.
	\hfill$\smash{∑¬_{i=j+1}^ℓ}H(\W i)=H(U_{j+1}^ℓ|U₁^jY₁^ℓ)$.
}

\frame{{Local CLT behavior 3/4}
	What is $H(U_{j+1}^ℓ|U₁^jY₁^ℓ)$?	\hfill($j≔⌈H(W)+ℓ^{-1/2+α}⌉$)
	\pp
	It is the conditional entropy of	\\
	noisy-channel coding.
	\tikz[x=1em,y=-1em,overlay,shift={(9,-2)}]{
		\draw
			(0,.5)rectangle node(G){$·G$}(3,5.5)
			foreach\k in{1,...,5}{
				(-2,\k)node(U\k)[opacity={\k<3?0:1}]{$U_\k$}(U\k)-|(0,3)
				(5,\k)node(X\k){$X_\k$}(3,3)|-(X\k)
				(8,\k)node(Y\k){$Y_\k$}(X\k)--(Y\k)
			}
		;
	}
	\pp
	Gallager has good bounds.
}

\frame{{Local CLT behavior 4/4}
	The last segment: $∑¬_{i=1}^{j}h(H(\W i))<4ℓ^{1/2+α}$.	\\
	Pre-process by Jensen inequality: $jh(÷1j∑¬_{i=1}^{j+1}H(\W i))<4ℓ^{1/2+α}$.	\\
	Chain rule: $jh(÷1jH(U₁^j|Y₁^ℓ))$, but what is $H(U₁^j|Y₁^ℓ)$?
	\pp
	Wiretap channel [new idea];	\\
	Hayashi has good bounds.
	\tikz[x=1em,y=-1em,overlay,shift={(9,-4)}]{
		\def\comment{\llap{\ifnum\k<3 message \else obscure \fi}}
		\draw
			(0,.5)rectangle node(G){$·G$}(3,5.5)
			foreach\k in{1,...,5}{
				(-2,\k)node(U\k){\comment$U_\k$}(U\k)-|(0,3)
				(5,\k)node(X\k){$X_\k$}(3,3)|-(X\k)
				(8,\k)node(Y\k){$Y_\k$}(X\k)--(Y\k)
			}
		;
	}
}

\frame{{A calculus machinery [new idea]}
	Given local LDP behavior: $Z(\W j)≤ℓe^{qZ(W)ℓ}(qZ(W))^{⌈j^2/3ℓ⌉}$	\\
	and local CLT behavior: $∑¬_{j=1}^ℓh(H(\W j))<4ℓ^{1/2+α}$.
	\pp
	eigen: $𝘌[h(𝘏_{n+1})|𝘏₀,…,𝘏_n]≤ℓ^{-1/2+3α}h(𝘏_n)$.
	\pp
	en23: $𝘗{𝘡_n<e^{-n^{2/3}}}>1-H(W)-ℓ^{(-1/2+4α)n}$.
	\pp
	een13: $𝘗『𝘡_n<\exp\(-e^{n^{1/3}}\)』>1-H(W)-ℓ^{(-1/2+4α)n}$.
	\pp
	elpin: $𝘗{𝘡_n<e^{-ℓ^{πn}}}>1-H(W)-ℓ^{-ρn}$.
}

\frame{{Summary of the proof}
	For local LDP behavior, we investigate the distance of a random matrix.
	\pp
	For local CLT, noisy-channel coding and wiretap-channel coding.
	\pp
	For the global MDP behavior, a calculus machinery is invented/used.
}

\frame{{Summary of my results so far}
	For all $π+2ρ<1$, there exist codes with	\\
	error probability $\P<e^{-N^π}$	and code rate $R>C-N^{-ρ}$.
	\pp
	When only $2×2$ kernels are allowed, at least $π,ρ>0$.
	\pp
	It happens that they have complexity $O(㏒N)$ per bit.
	\pp
	Can we reduce the complexity further	\\
	(at the expense of worse performance etc)?
}

\frame{{Prune the tree for simplicity}
	The bottom channel is good enough before we reach our favorite $n$.
	
	\onslide<+>{}
	$$\tikz{
		\path(0,0)circle(2pt)(0,4)circle(2pt);
		\channeltree0<6:8v49v128;
		\onslide<.(2)->{\draw[alerted text.fg](\insertoverlaynumber-1,1/4)--(7,1/4)
			node[right]{threshold $θ$};}
	}$$
	
	\onslide<.(2)->{Why do we apply transform any further? (Answer: we don't!)}
}

\frame{{Prune the other side}
	Sometimes, the top channel is too bad.	\\
	Do we expect any of its descendants to be good enough?
	
	\onslide<+>{}
	$$\tikz{
		\path(0,0)circle(2pt)(0,4)circle(2pt);
		\channeltree0<6:8v49v120;
		\onslide<.(3)->{\draw[example text.fg](\insertoverlaynumber-1,4-1/4)--(7,4-1/4)
			node[right]{threshold $1-θ$};}
		\onslide<.(2)->{\draw[alerted text.fg](\insertoverlaynumber-1,1/4)--(7,1/4)
			node[right]{threshold $θ$};}
	}$$
	
	\onslide<.(3)->{We don't.}
}

\frame{{Stopping time analysis}
	$𝘞_n$ has children/needs further transformation if $θ<𝘏_n<1-θ$.
	\pp
	Set $θ=N^{-10}$; assume $m>O(㏒(㏒N))$, then $e^{-2^{πm}}<θ$.
	\pp
	Then $𝘗\{𝘏_m<θ\}>𝘗{𝘏_m<e^{-2^{πm}}}≥H(W)-ℓ^{-ρm}$
	and $𝘗\{𝘏_m>1-θ\}>𝘗{H_m>1-e^{-2^{πm}}}≥1-H(W)-ℓ^{-ρm}$
	\pp
	That is to say, $𝘗\{θ<𝘏_m<1-θ\}≤2ℓ^{-ρm}$, hard to stay in the middle.
}

\frame{{Geometric complexity}
	Complexity $=$ \#transformations $=∑¬_{m=0}^n𝘗\{θ<𝘏_m<1-θ\}$.
	\pp
	$$\begin{cases}
		∑¬_{m=O(㏒(㏒N))}^n𝘗\{θ<𝘏_m<1-θ\}≤∑¬_{m=O(㏒(㏒N))}^n2ℓ^{-ρm}=O(1),	\\
		∑¬_{m=0}^{O(㏒(㏒N))}𝘗\{θ<𝘏_m<1-θ\}≤∑¬_{m=0}^{O(㏒(㏒N))}1=O(㏒(㏒N)).
	\end{cases}$$
	\pp
	Complexity is $O(㏒(㏒N))$ per bit, or $O(N㏒(㏒N))$ per block.
}

\frame{{Summary of pruning and whatnot}
	\pp
	There exist codes with complexity $O(㏒(㏒N))$ per bit,	\\
	error probability $\P<N^{-9}$, and code rate $R=C-N^{-ρ}$.
	\pp
	(Earlier) we have codes with complexity $O(㏒N)$ per bit,	\\
	error probability $\P<e^{-N^π}$, and code rate $R>C-N^{-ρ}$.
	\pp
	Are there codes in between? Yes, continuously.
}

\frame{{Summary}
	Log-log code taken from (with Duursma)	\\
	Log-logarithmic Time Pruned Polar Coding	\\
	\url{https://ieeexplore.ieee.org/document/9274497}.
	
	MDP code taken from (with Duursma)	\\
	Polar Codes' Simplicity, Random Codes' Durability	\\
	\url{https://ieeexplore.ieee.org/document/9274521}.
}

\frame{{Question?}
	\inserttitlegraphic
	
	Predefined questions:	\\
	What does each chapter in dissertation do?	\\
	Why input alphabet is finite field? What is the  advantage?	\\
	Definition of Bhattacharyya parameter?	\\
	References for XYZ?	\\
	Your contribution over others?	\\
	Future plan?
}

\def\appendixname{Appendix}
\appendix

\pgfplotstableread{
		Code				Error			Gap			Complexity		Channel		
		random				e^{-N^π}		N^{-ρ}		\exp(N)			DMC			
		concatenation		e^{-N^π}		→0			\poly(N)		DMC			
		RM					→0				→0			O(N^2)			BEC			
		LDPC				→0				→0			†unclear†		SBDMC		
		{RA family}			→0				→0			O(1)			BEC			
		old~prune			e^{-N^{1/2}}	O(1)		Θ(㏒N)			SBDMC		
		{loglog-polar [W.]}	e^{-n^τ}		N^{-ρ}		O(㏒(㏒N))		DMC			
		{MDP-polar [W.]}	e^{-N^π}		N^{-ρ}		O(㏒N)			DMC			
}\tableComplex
\pgfplotstablemodifyeachcolumnelement{Error}\of\tableComplex\as\cell
	{\edef\cell{$\unexpanded\expandafter{\cell}$}}
\pgfplotstablemodifyeachcolumnelement{Gap}\of\tableComplex\as\cell
	{\edef\cell{$\unexpanded\expandafter{\cell}$}}
\pgfplotstablemodifyeachcolumnelement{Complexity}\of\tableComplex\as\cell
	{\edef\cell{$\unexpanded\expandafter{\cell}$}}
\frame{
	$$\pgfplotstabletypeset\tableComplex$$
}

\def\decodecontent#1#2\relax{
	\if\pgfplotstablecol0	\assigncontent{#1#2}
	\else\if#1w				\assigncontent{[W.]}
	\else					\assigncontent{\footnotesize\cite{#1#2}}
	\fi\fi
}
\frame{
	$$\pgfplotstabletypeset[
		every head row/.style={
			before row=\toprule&\multicolumn5c{Symmetric}&\multicolumn2c{Asymmetric}\\,
			after row=\midrule},
		assign cell content/.code={\decodecontent####1\relax}
	]\tableRefarray$$
}

激®{\color{alerted text.fg}}
\frame{{Input alphabet [new idea]}
	$\begin{bmatrix}
		W(y₁|1)		&	W(y₂|1)		&	W(y₃|1)		&	⋯	\\
		W(y₁|2)		&	W(y₂|2)		&	W(y₃|2)		&	⋯	\\
		W(y₁|3)		&	W(y₂|3)		&	W(y₃|3)		&	⋯	\\
		W(y₁|4)		&	W(y₂|4)		&	W(y₃|4)		&	⋯	\\
		W(y₁|5)		&	W(y₂|5)		&	W(y₃|5)		&	⋯	\\
		W(y₁|6)		&	W(y₂|6)		&	W(y₃|6)		&	⋯	\\
		®W(y₁|6)	&	®W(y₂|6)	&	®W(y₃|6)	&	®⋯	\\
	\end{bmatrix}$
	\qquad
	\tikz[x=4em,y=-2em,baseline=-8em]{
		\draw[line width=1em+2*rule_thickness,postaction={draw=bg,line width=1em}]
			(-1,1)--(-1,7)(0,1)--(0,6)(1,1)--(1,6);
		\fill
			foreach\y in{1,...,7}{(-1,\y)circle(2pt)}
			foreach\y in{1,...,6}{(0,\y)circle(2pt)}
			foreach\y in{1,...,6}{(1,\y)circle(2pt)}
		;
		\draw[shorten <=3pt,shorten >=3pt]
			foreach\y in{1,...,6}{(-1,\y)edge[->](0,\y)}
			(-1,7)edge[->,alerted text.fg](0,6)
			foreach\y in{1,...,6}{
				(0,\y)edge[->](1,1+rnd*5)edge[->](1,1+rnd*5)
			}
		;
	}
}

\frame{{Asymmetric channels \cite{HY13}}
	Recall $U_j$ is the coordinate as in $X₁^ℓ≔U₁^ℓ·G$.	\\
	The difficulty of asymmetric channels is $U_j$ being nonuniform and dependent.
	
	Define synthetic channel $\Q i≔(U_i|U_1^{i-1})$.	\\
	Define tree $\Q i,\QQ ij,\QQQ ijk,…$; define channel process $\{𝘘_n\}$.	\\
	It polarizes, and at the same pace.
	
	High $H(𝘘_n)$ low $H(𝘞_n)$ vs both high vs both low.
}

\frame{{Bhattacharyya parameter}
	Binary $Z(W)≔∑¬_{y∈𝒴}√{W(y|0)W(y|1)}$.
	
	Non-binary
	$≔÷1{q-1}∑¬_{\substack{x,x'∈𝔽_q\\x≠x'}}∑¬_{y∈𝒴}√{W(x,y)W(x',y)}$. \\
	{}[New idea] $≔\max¬_{0≠d∈𝔽_q}∑¬_{x∈𝔽_q}∑¬_{y∈𝒴}√{W(x,y)W(x+d,y)}.$
}

\frame{{A List of Important Contributions (Chronological)}
	A combinatorial trick to recover scaling exponent (een13 $→$ elpin).
	
	The pruning technique/stopping time analysis/log-log complexity.
	
	Improved combinatorial trick (en23 $→$ een13 $→$ elpin).
	
	Dynamic kernel, later random dynamic kernel.
	
	Alphabet reduction to finite field (trivial but powerful and last mile).
	
	Improve definition of Bhattacharyya parameter, then and FTPC$Z$.
	
	Reducing local CLT to noisy-channel and wiretap-channel coding.
	
	For wiretap bound, extend the universal bound via continuation.
	
	A topological argument to show positive $ϱ$ (that is, CLT⋆).
}

%%%% Bi-Hölder toll

\tiny
\advance\lineskip0ptplus1em
\advance\baselineskip0ptplus1em
\setbeamertemplate{bibliography item}[text]
\setbeamertemplate{bibliography entry author}{\bgroup}
\setbeamercolor{bibliography entry author}{fg=alerted text.fg}
\setbeamercolor{bibliography entry location}{fg=normal text.fg}
\setbeamertemplate{bibliography entry note}{\egroup}
\bibliographystyle{alphaurl}
\bibliography{Chilly-beamer}
\vfil\hbox{}

\frame{}

\end{document}




